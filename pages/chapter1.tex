% Chapter1

\chapter{引言}
有效市场假说(Efficient Markets Hypothesis,EMH)是现代金融学的理论基石之一。Fama(1965)首次提出了“有效市场”的概念,并定义为:如果在一个证券市场中,价格完全反映了所有可以获得的讯息,那么就称这样的市场为有效市场。他认为市场上存在着大量理性的投资者,在任何时间都有无数人正在搜寻细微的线索去精准预测股票未来的价格,为了自身利益去对股票进行低买高卖,或许对于个人而言仅仅是超额利润的攫取,但在宏观维度上,正是这种高量级的活动快速推进着股票市场价格向正确价格的趋近,任何相关的信息已经完全展现在了股票价格中,所有为了预测付出的时间、金钱和努力都是徒劳无功。进而Fama(1970)在《Journal of Finance》杂志上发表了最具划时代意义的论文《Efficient Capital Markets: A Review of Theory and Empirical Work》,正式提出了有效市场假说,并对证券市场的信息分为以下三类:一是与交易相关的历史信息,例如历史成交量、价格等;二是当前的公开信息,如公司财报、分红报告等;三是内部信息。Fama按照以上三种信息把有效市场分为弱式有效、半强式有效和强式有效三类,其中弱式有效证明了技术分析的无效性;半强式有效排除了基本面分析的作用;而强式有效意味着股票价格已经反映了所有信息,投资者不可能战胜市场。至此确立了“有效市场假说”在现代金融领域的基础性地位。

但是现在越来越多的实证研究表明了中国股票市场的非有效性。自70年代开始,随着科技的发展,持续识别出能够提供超额收益的异象(因子)。曾有研究发现了股票价格过度波动和可预测性(Shiller,1981)、股票价格走势逆转(Shieifer、Vishney,1994)等异常情况;在单只股票或投资组合领域,也发现了规模效应(Bamber等,1997)、市盈率效应(Banz,1981)等,前美国金融学会主席Cochrane(2011)称其为“因子动物园”(Factor Zoo),这些异象说明投资者能够基于基本面等公开信息去预测股票价格,向有效市场假说发起了挑战。

另外,从行为金融角度看,很多学者指出有效市场假说理论的逻辑存在问题。如果股票价格已经完全反映市场中的信息,那理性投资者在后续的投资活动中就无需花费精力去搜集信息、预测股价了,但这却与市场需要信息才能有效运行相悖。同样,其设定的完美市场假定条件脱离实际。在实际的投资活动中,投资者更多表现为“非理性”,从而导致其决策行为偏离金融理论预测的标准结果,例如羊群效应、月末效应、规模效应和股权溢价之谜等等金融异象都是投资者“非理性”行为的表现(张元鹏,2015)。虽然行为金融文献并没有完全否定有效市场的界定,但行为金融派的支持者更愿将有效市场看作一种理想状态、不受个人意志(效用和偏好)影响的市场状态(丁志国等,2017)。

从以上几个方面来看,我国股票市场未能做到完全有效,虽然A股市场现已发展成为全球第二大股票市场,总市值近80万亿元。但由于成立时间晚、相关制度不完善、散户比例大、交易成本高等问题,导致了定价效率的低下。这时基本面分析更能捕捉市场的非有效性,有效预测未来盈余,带来显著的超额回报(汪荣飞和张然,2018)。相对于技术分析依靠图表去预测价格趋势,基本面分析更为关注公司的基本信息,主要包括财务报表或非财务上的信息,并从中评估公司的内在价值。本文将股票价格与内在价格的偏离定义为错误定价,而这种错误定价便是投资者所追捧的超额收益,所以对于错误定价的研究显得极为重要。

由于股票价格是已知的,那么对错误定价的研究关键在于对股票内在价格、公司内在价值的预测。现有研究中常见的计算方法包括现金流贴现法(Free Cash Flow Valuation,FCFF);相对价值法,即与同行业的相似公司对比,以它们的平均市盈率、市净率及市销率等指标来计算;经济附加值法(Economic Value Added,EVA);实物期权法等。这些方法虽然起到了一定的预测作用,但由于其选择的指标过少,或太过于依赖对未来的主观预测结果,亦或计算方法过于程式化等原因,很难排除数据窥视(Data snooping)的影响。

基于以上分析,本文创新采用大数据统计方法计算公司的内在价值。利用2003年初到2020年底的季度财务报表数据,对A股所有公司(剔除金融类股票、ST股票)进行时间维度的截面回归:公司市场价值对于一系列财务指标的线性回归,这些财务指标的组合结果即为公司内在价值,回归残差为公司市场价值与内在价值的偏差,将偏差按照公司市场价值标准化后,成功构建出错误定价因子M(Mispricing Factor)。其中,线性回归系数按时点分组截面计算,避免了该时点下其他投资因素的影响;纳入A股所有公司,回归结果更具有普遍性;且后续模型修订根据客观统计标准,排除了数据窥视的影响。

在成功引入错误定价因子M后,接下来便是检验其有效性。问题在于:一、能否通过错误定价因子M构建低买高卖的投资组合获取超额收益,进而证明基本面分析的可行性?二、超额利润来自于错误定价还是因子遗漏,股票市场价格是否会向内在价格趋近?本文将针对以上两个问题进行研究。首先从CSMAR上搜集A股上市公司季度财务数据,计算每个时点公司的内在价值和错误定价因子M;后采用Fama-MacBeth回归计算错误定价因子M系数$\beta$,检验其是否显著;并且按照错误定价因子M将公司分为五组,从价值最被高估到最被低估,运用因子模型得超额收益$\alpha$,观察是否呈现单增趋势;最后观察长期时间内不同分组超额收益的变动趋势,证明超额收益并非来源于风险因子的遗漏。最后结论证明了错误定价因子M的有效性,有力支持了基本面分析之于中国股市证券分析的地位。

本文的研究有一定的现实意义与理论贡献:一、丰富了经济学和管理学研究的工具箱;二、丰富了基本面投资的理论和实践研究;三、丰富了中国股票市场有效性的研究。目前对于中国A股市场基本面分析的研究过少且不完善,以大数据统计方法的研究更为稀有,统计之于数据分析的重要性不言而喻,其能够有效避免数据窥视的影响,是在此研究方向上的创新。同样,建模方法也有所创新,在线性回归的基础上增加了非线性维度的考量。

后文结构如下:第二部分回顾了中国股市有效性、基本面分析等相关文献,第三部分阐述了本文的研究设计,介绍数据和变量的构建方法;第四部分详细分析了本文的实证结果,第五部分为本文的研究结论。
