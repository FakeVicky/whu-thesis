% Chapter1

\chapter{引言}
有效市场假说(Efficient Markets Hypothesis,EMH)是现代金融学的理论基石之一。~\citeauthor{famaBehaviorStockMarketPrices1965}(\citeyear{famaBehaviorStockMarketPrices1965})首次提出了“有效市场”的概念,并定义为:如果在一个证券市场中,价格已经完全反映了所有可以获得的信息,那么就称这样的市场为有效市场。他认为市场上存在着大量理性的投资者,在任意时点都有无数人正在搜寻线索去精准预测股票未来的价格,为了获取利益而去对股票进行低买高卖,或许对于个人而言仅仅是超额利润的攫取,但在宏观维度上,正是这种高量级的同质活动快速推进着股票市场价格向正确价格的趋近,所有相关的信息已经完全展现在了股票价格中,任何为了预测股价而付出的时间、金钱和努力都是徒劳无功。进而~\citeauthor{famaEfficientCapitalMarkets1970}(\citeyear{famaEfficientCapitalMarkets1970})在《Journal of Finance》杂志上发表了最具划时代意义的论文《Efficient Capital Markets: A Review of Theory and Empirical Work》,自此正式提出了有效市场假说,并对证券市场的信息分为以下三类:一是与交易相关的历史信息,例如历史成交量、价格等;二是当前的公开信息,如公司财报、分红报告等;三是内部信息。Fama按照以上三种信息把有效市场分为弱式有效、半强式有效和强式有效三类,其中弱式有效证明了技术分析的无效性;半强式有效排除了基本面分析的作用;而强式有效意味着股票价格已经反映了所有信息,投资者不可能战胜市场。至此确立了“有效市场假说”在现代金融领域的基础性地位。

但是现在越来越多的实证研究表明了中国股票市场的非有效性。自70年代开始,随着科技的发展,已经持续识别出了能够提供超额收益的异象(因子)。曾有研究发现了股票的一些异常情况,例如价格的过度波动和可预测性(\citeauthor{shillerStockPricesMove1980},\citeyear{shillerStockPricesMove1980})、价格走势逆转(\citeauthor{shleiferPoliticiansFirms1994},\citeyear{shleiferPoliticiansFirms1994})等;在投资组合领域,也发现了规模效应(\citeauthor{bamberTradingVolumeDifferent1997},\citeyear{bamberTradingVolumeDifferent1997})、市盈率效应(\citeauthor{banzRelationshipReturnMarket1981},\citeyear{banzRelationshipReturnMarket1981})等,前美国金融学会主席~\citeauthor{cochranePresidentialAddressDiscount2011}(\citeyear{cochranePresidentialAddressDiscount2011})称其为“因子动物园”(Factor Zoo),这些异象说明投资者能够根据基本面等公开信息去预测股票价格,向有效市场假说发起了挑战。

另外,从行为金融角度看,有很多学者指出有效市场假说理论的逻辑存在问题:如果股票价格已经完全反映市场中的信息,那理性投资者在后续的投资活动中就无需花费精力去搜集信息、预测股价了,但这却与市场需要信息才能有效运行相悖。同样,其设置的完美市场假设的条件与实际相悖,在真实的投资活动中,投资者较多地表现为“非理性”,从而使得其决策行为偏离了金融理论所要求和预测的标准结果,例如羊群效应、月末效应、规模效应和股权溢价之谜等等这些金融异象都被认为是投资者非理性决策行为的主要表现(\citeauthor{zhangTouZiZheZhenDeShiLiXingDeMaXingWeiJinRongXueDuiFaMaDeShiChangYouXiaoJiaShuo2015e},\citeyear{zhangTouZiZheZhenDeShiLiXingDeMaXingWeiJinRongXueDuiFaMaDeShiChangYouXiaoJiaShuo2015e})。虽然行为金融文献并没有完全否定有效市场的界定,但行为金融派的支持者更愿将有效市场看作一种理想状态、不受个人意志(效用和偏好)影响的市场状态(\citeauthor{dingYouXiaoShiChangDeJianYanXingWeiJinRongDuiEMHLiLunDePiPan2017},\citeyear{dingYouXiaoShiChangDeJianYanXingWeiJinRongDuiEMHLiLunDePiPan2017})。

从以上几个方面来看,我国股票市场未能做到完全有效,虽然A股市场现已发展成为全球第二大股票市场,总市值近80万亿元。但由于成立时间晚、相关制度不完善、散户比例大、交易成本高等问题,导致了定价效率的低下。这时基本面分析更能捕捉市场的非有效性,有效预测未来盈余,带来显著的超额回报(\citeauthor{wangJiBenMianFenXiZaiZhongGuoAGuShiChangYouYongMaLaiZiJiDuCaiWuBaoBiaoDeZhengJu2018},\citeyear{wangJiBenMianFenXiZaiZhongGuoAGuShiChangYouYongMaLaiZiJiDuCaiWuBaoBiaoDeZhengJu2018})。相对于技术分析依靠图表去预测价格趋势,基本面分析更为关注公司的基本信息,主要包括财务报表或非财务上的信息,并从中评估公司的内在价值。本文将股票价格与内在价格的偏离定义为错误定价,而这种错误定价便是投资者所追捧的超额收益,所以对于错误定价的研究显得极为重要。

由于股票价格是已知的,那么对错误定价的研究关键在于对股票内在价格、公司内在价值的预测。现有研究中常见的计算方法包括现金流贴现法(Free Cash Flow Valuation,FCFF);相对价值法,即与同行业的相似公司对比,以它们的平均市盈率、市净率及市销率等指标来计算;经济附加值法(Economic Value Added,EVA);实物期权法等。这些方法虽然起到了一定的预测作用,但由于其选择的指标过少,或太过于依赖对未来的主观预测结果,亦或计算方法过于程式化等原因,很难排除数据窥视(Data snooping)的影响。

近年来,“金融科技”概念异军突起,银行需要和互联网金融巨头进行竞争,只能借助科技进行转型,金融服务逐渐“线上化”;市场要突破传统的金融方式,实现信息安全,需要区块链技术等进行加密等等,金融与IT技术的结合已然成为未来的大趋势。
而机器学习是计算机科学领域重要的基础学科之一,主要通过人工智能方法模拟人类的行为,去对数据进行学习,并在一次又一次的重复试验中累积经验,逐渐优化算法,让结果更具有效性。
随着科学技术的高速发展,计算机的计算效率显著提升,其所能处理的数据以及算法越来越多。
机器学习因其能够突破人类计算能力的特点,重新向大家定义了科学研究的深度和广度,不仅局限于计算机领域,各行各业均需要借助其力量进行更为复杂的分析活动。
在金融领域,随着中国证券市场的发展、投资者行为的多样化等因素,已很难去进行大规模的金融数据分析,而机器学习则可以做到这一点。

在学术研究中,近年来越来越多学者运用机器学习方法,去关注金融市场预测、资产定价等问题,并取得了丰厚的成果;在现实交易中,量化投资者以数据模型为核心,以程序化交易为手段,以追求绝对收益为目标进行投资,大多采用机器学习算法去实现交易思想。

相比于以往金融研究中运用的传统计量方法,机器学习方法有着以下优势:
\begin{itemize}
\item 对比于传统算法,机器学习方法是一种通过模仿人类的学习行为,逐渐累积经验的过程,每次学习过程中,数据预测能力的提升均来自于客观标准,而非人工干预,避免了数据窥视的影响,对于实证结果更具有可解释性。
\item 机器学习方法对于数据的接受度更高,在各路专家的集中努力下,机器学习模型有了多个实现算法,且愈发优化,能够高效处理维度更大、结构更复杂的数据,甚至对于以往难以处理的定性文本数据都能简单处理。
\item 金融数据适合用机器学习方法处理,金融市场也多表现为动态系统。例如在时间序列数据中,大多数金融属性并不符合线性关系,反而更多呈现为非线性、高维度和噪声性质,很难通过传统的计量方法去解决\cite{zhaoJiQiXueXiZaiJinRongZiChanJieGeYuCeHePeiZhiZhongDeYingYongYanJiuShuPing2020}。但机器学习算法高度灵活,不会预先在主观上对函数形式、变量分布等条件进行假设。而且,很多有关价格预测、资产定价的金融问题,都是涉及到对历史数据的分析,并在准确预测的基础上构建投资组合,数据量较大。
\end{itemize}

同样,在现有的基本面分析研究中,因子的选择较少,交易策略较为局限。
为了突破传统因子研究的局限性,本文创新采用财务指标去构建错误定价因子$M$,并以此进行资产定价与配置。该因子的构建不参杂任何的主观因素,均通过客观标准去进行优化选择。

基于以上分析,本文采用机器学习方法——提升回归树BRTs(Boosted Regression Trees)统计预测公司的内在价值。利用2002年初到2019年底的季度财务报表数据,对A股所有公司(剔除金融类股票、ST股票)进行时间维度的截面回归:公司市场价值对于一系列财务指标进行非参数回归,这些财务指标的组合预测结果即为公司内在价值,回归残差为公司市场价值与内在价值的偏差,将偏差按照公司市场价值标准化后,成功构建出错误定价因子$M$(Mispricing Factor)。其中,回归系数将分期进行计算,避免了该期其他投资因素的影响;纳入A股所有公司,使回归结果更具有普遍性;且后续模型修订根据客观统计标准,排除了数据窥视的影响。

在成功引入错误定价因子$M$后,接下来便是检验其有效性。问题在于:一、能否通过错误定价因子$M$构建低买高卖的投资组合获取超额收益,进而证明基本面分析的可行性?二、超额利润来自于错误定价还是因子遗漏,股票市场价格是否会向内在价格趋近?本文将针对以上两个问题进行研究。首先从CSMAR上搜集A股上市公司季度财务数据,通过BRTs算法计算每期公司的内在价值和错误定价因子$M$;后采用Fama-MacBeth回归计算错误定价因子$M$系数$\beta$,检验其是否显著;并且按照错误定价因子$M$将公司分为五组,从价值最被高估到最被低估,运用因子模型得超额收益$\alpha$,观察是否呈现单增趋势,以及构建多空投资组合,检验是否有显著为正的超额收益;最后观察长期时间内不同分组超额收益的变动趋势,证明超额收益并非来源于风险因子的遗漏。最后结论证明了错误定价因子$M$的有效性,有力支持了基本面分析之于中国股市证券分析的地位。

本文的研究有一定的现实意义与理论贡献:一、丰富了经济学和管理学领域的研究内容;二、丰富了基本面投资的理论和实践研究;三、丰富了中国股票市场有效性的研究。目前对于中国A股市场基本面分析的研究过少且不完善,统计之于数据分析的重要性不言而喻,其能够有效避免数据窥视的影响,是在此研究方向上的创新。同样,建模方法也有所创新,在线性回归的基础上增加了非线性维度的考量,采用机器学习算法去进行数据拟合,更具有信服度。

后文结构如下:第二部分回顾了中国股市有效性、基本面分析等相关文献,第三部分对本文采用的机器学习模型BRTs进行详细介绍,第四部分阐述本文的研究设计,介绍了数据和变量的构建方法;第五部分详细分析了本文的实证结果,第六部分为本文的研究结论。
