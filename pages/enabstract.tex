% 英文摘要

In traditional fundamental analysis, we have to subjectively predict the intrinsic price of stocks, the calculation methods include Free Cash Flow Valuation (FCFF), relative value assessment, EVA, real option method, etc. Although these methods have played a certain role in forecasting, it is difficult to rule out the influence of data snooping due to the selection of too few indicators, or too much dependence on the subjective prediction results of the future, or the stylized calculation method. Also, there are too few factors to choose in the current research that it is difficult to innovate investment strategies.

Compared to traditional fundamental analysis, we innovatively use statistical model based on big data. Focus on the quarterly financial statements data of 3,300 stocks in China's A-share market from the beginning of year 2002 to the end of year 2019, 51 financial indicators are selected to predict on the company's intrinsic value using boosted regression trees (BRTs) in machine learning. Compared to linear regression, machine learning methods can effectively process high-dimensional factor data and can dig out the nonlinear relationship between variables, which performs well in existing applications.

We define the deviation between the company's market value and the intrinsic value as a mispricing factor, and classify companies according to it, finally construct the corresponding portfolio accordingly. The empirical results show that the long position and the long-short portfolio constructed by this above strategy can obtain significant positive monthly average returns and risk-adjusted returns, which proves the effectiveness of using the mispricing factor for asset pricing and allocation. The follow-up robust test reflects the gradual approach of the stock market value to the intrinsic value, which proves that the excess return does not come from the omission of risk factors. 

The conclusions of this article strongly support the status of fundamental analysis in China's stock market and securities analysis, and promote the intersection of machine learning and economic research.