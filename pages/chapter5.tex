% Chapter5

\chapter{结论与启示}
本文基于2003-2020年的季度财务报表数据,选取14个缺失值比例低、共线性程度小的财务指标,以线性回归的形式完成了对公司内在价值的预测,并以估计的内在价值与市场价值的差异构建了错误定价因子M,当M为负值时,说明公司价值最被高估,当M为正值时,说明公司价值最被低估。根据错误定价因子M对公司进行分类,并进行相应的高卖低买策略,研究证明从最被高估到最被低估的投资组合,超额收益$\alpha$呈现显著的递增关系,并且其多空组合能持续5年左右的正收益。同样也强调了新旧信息迭代的重要性,需经常性对数据进行重新调整与分析。以上结论在Fama-MacBeth截面回归和季度投资组合的五种因子模型检验中均显著符合预期,能够通过未来28期的稳健性测试,具有可观的经济意义。综上所述,本文的实证结果表明,基于季度财务报表的统计分析能够有效运用于中国A股市场,基本面分析起到了一定的预测作用。

基本面分析的有效性也反映出我国资本市场的效率低下,虽然我国A股市场总市值近80万亿元,已发展为全球第二大股票市场。但仍然由于成立时间晚、相关制度不完善、散户比例大、交易成本高等问题,导致了定价效率的低下。而市场效率的提高需要从多方面努力。
\begin{itemize}
\item[(1)] 投资者对于市场信息应仔细甄别,避免盲目跟风,根据自身的风险承担水平构建适合自己的投资组合;提高认知水平,学习相关财务知识,能够做到财务报表的简单阅读,市场基本面信息的理解。

\item[(2)] 监管部门应加强建立与完善监管披露机制,让信息传递更具透明性与高效率。在现有科技社会,推进市场数据的信息化,利用大数据技术做到数据的及时披露,及时筛选并纠正市场上的错误信息,引导投资者理性投资;相反,该举措也能做到信息的向上反馈,让政策制定者更为清晰的了解到市场的实际运行情况,为国家宏观政策的制定提供依据。但同时也要做到对用户个人隐私的保护。

\item[(3)] 上市公司需严格遵守会计规则,做到财务报表的及时、准确、公平披露,不得误导投资者,并做好内幕信息的知情人登记工作,减少投资者层面上的信息不对称。


\end{itemize}


