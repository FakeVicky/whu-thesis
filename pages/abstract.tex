% 中文摘要

传统的基本面分析方法较为受限。一方面,需要主观预测股票内在价格,其常见计算方法包括现金流贴现法、相对价值法、经济附加值法、实物期权法等。这些方法虽然起到了一定的预测作用,但由于选择的指标过少,或太过于依赖对未来的主观预测结果,亦或计算方法过于程式化等原因,很难排除数据窥视的影响。另一方面,目前研究中可供选择的因子种类较少,难以做到投资策略的推陈出新。

相较而言,本文创新采用大数据统计方法,基于中国A股市场3300支股票自2002年初到2019年底的季度财务报表数据,选取51个财务指标,采用机器学习中的提升回归树BRTs算法对公司内在价值进行回归预测。对比于传统的线性回归方法,机器学习方法能够有效处理高维度的因子数据,挖掘出变量之间的非线性关系,在现有应用中表现良好。

本文将公司市场价值与内在价值的偏差构建为错误定价因子,根据错误定价因子将公司进行分类,据此构建相应的投资组合。实证结果表明,该投资策略构建的多头和多空投资组合能够获得显著的正月度平均收益和风险调节收益,证明了运用错误定价因子进行资产定价与配置的有效性。后续稳健型检验反映了股票市场价值向内在价值的趋近,证明超额收益并非来源于风险因子的遗漏。

本文结论有力支持了基本面分析之于中国股市证券分析的地位,并促进了机器学习与经济学研究的交叉融合。
