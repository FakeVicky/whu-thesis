% Chapter 2

\chapter{文献综述}
对于中国股票市场,已有多位学者研究证明其不具有弱势有效性。
贾权和陈章武(2003)对基于市场有效假设的CAPM模型以及其他因素与收益率之间的关系进行了实证检验,发现目前我国股市不满足市场有效性的假设,投资者的行为并不是完全理性的;
吴振翔和陈敏(2007)通过设计多种投资组合方式,发现在短期(3个月以内)不能否定市场中无套利的假定;而对于中期和长期(6个月及12个月) 统计套利存在,说明我国A股股票市场的弱有效性不成立;
Lim和Brooks(2009)运用非线性依赖检验的方法对1999年至2005年上证A股和B股的日数据进行检验,发现A股和B股均不具备弱势有效性;
屈博和庞金风(2014)利用Q统计量法、方差比检验法、广义谱检验、游程检验等多种方法对2010年至2013年沪深300股指期货的当月合约的5分钟高频数据进行检验,实证结果表明,随着考察期长度的增加,市场趋于拒绝弱式有效性。

上述研究说明,中国股市能够通过技术分析和基本面分析获取超额收益,实际上很
多实证研究也证明了这个结论。
对于技术分析,赵国顺(2009)基于时间序列方法,使用GARCH模型和ARIMA模型对股价波动趋势有着较好的短期预测效果;
王劲松(2010)运用技术指标MA、KDJ和MACD证明在一定时间跨度上技术分析是有效的;
石赛男(2011)的实证研究表明MACD指标对于2004年至2009年内的大、中、小盘股均有一定的预测能力;
包懿(2015)基于2010到2014年三只指数标的和一只个股标的,通过持续持有策略和均线穿越法则策略获取了超额收益。
对于基本面分析,张然等(2017)采用日历时间组合的方法,证实中国A股市场的分析师修正信息具有投资价值,并且这个投资价值主要来源于其能够预测公司基本面信息。
汪荣飞和张然(2018)基于季度财务指标构建了六组基本面指标,发现均能够有效预测未来盈余,进而发现分析师和投资者均没有意识到基本面指标的价值。
常丹婷和李峰(2020)构建了价值因子、基本面两个维度进行横截面分析,发现只有当估值和基本面预期背离、存在错误定价时,高低估值的对冲组合才能产生显著的超额收益,高达16.8\%。

目前国内有关基本面分析的研究较少,但在成熟的国外资本市场,基本面分析的作用已被大量文献证实。
Ou和Penman(1989)构建了68组基本面指标,发现其能够成功预测未来股票收益涨跌的概率,多空组合在两年内达到12.5\%的超额收益。
Abarbanell和Bushee(1997)构建了12组基本面指标,研究发现大部分指标能够预测股票未来价值,Abarbanell和Bushee(1998)进一步利用这些基本面指标构建投资组合,获得了13.2\%的年化超额收益,并发现超额收益大部分与未来的业绩公告相关。
Piotroski(2000)对高账面市值比的公司进行了研究,并用9组基本面指标构建出综合指标FSCORE,用FSCORE挑选出真正的价值股并提升了7.5\%的收益率。
同样,Mohanram(2005)着眼于低账面市值比的成长股,利用8组基本面指标构建了综合指标GSCORE。
Asness等(2019)根据盈利性、成长性、安全性,构建了衡量公司质量的QMJ(quality minus junk)指标,发现高质量公司股价较高,但低质量公司不定,多头高质量公司空头低质量公司的投资组合信息比率大于1。

上述研究均根据基本面指标构建了综合指标,并且构建对应的多空组合获取了超额收益,证明了基本面分析的有效性。但国内有关基本面分析的研究较少,运用大数据统计方法的更少,所以本文将通过构建错误定价因子M来进行我国A股市场基本面分析可行性的研究。


%\section{公式的使用}
%在文中引用公式可以这么写:\(a^2 + b^2 = c^2\)。这是勾股定理,它还可以表示为 \(c = \sqrt{a^2 + b^2}\)。还可以让公式单独一段并且加上编号:
%注意,公式前请不要空行。
%
%还可以通过添加标签在正文中引用公式,如式\eqref{eq:pingfanghe}。
%
%我们还可以轻松打出一个漂亮的矩阵:
%\begin{equation}
%  \vb*{A} =
%  \begin{bmatrix}
%    1  & 2  & 3  & 4  \\
%    11 & 22 & 33 & 44 \\
%  \end{bmatrix} \times
%  \begin{bmatrix}
%    22 & 24 \\
%    32 & 34 \\
%    42 & 44 \\
%    52 & 54 \\
%  \end{bmatrix}
%\end{equation}
%
%或者多行对齐的公式:
%\begin{equation}
%  \begin{aligned}
%    f_1(x) & = (x + y)^2         \\
%           & = x^2 + 2 x y + y^2
%  \end{aligned}
%\end{equation}
%
%模板使用了 unicode-math 包更改数学字体。所以在使用数学字体时,尽量使用 unicode-math 包提供的 \verb|\sym| 接口,详情请阅读 unicode-math 文档。
%
%\section{插图的使用}
%\begin{figure}
%  \centering
%  \includegraphics[width=0.3\textwidth]{whulogo.pdf}
%  \caption{插图示例}
%  \label{fig:whu}
%\end{figure}
%
%\LaTeX{} 环境下可以使用常见的图片格式:JPEG、PNG、PDF 等。当然也可以使用 \LaTeX{} 直接绘制矢量图形,可以参考 pgf/ti\emph{k}z 等包中的相关内容。需要注意的是,无论采用什么方式绘制图形,首先考虑的是图片的清晰程度以及图片的可理解性,过于不清晰的图片将可能会浪费很多时间。
%
%\verb|[htbp]| 选项分别是此处、页顶、页底、独立一页。\verb|[width=\textwidth]| 让图片占满整行,或 \verb|[width=2cm]| 直接设置宽度。可以随时在文中进行引用,如图~\ref{fig:whu},建议缩放时保持图像的宽高比不变。
%
%如果一个图由两个或两个以上分图组成时,各分图分别以(a)、(b)、(c)...... 作为图序,并须有分图题。模板使用 subcaption 宏包来处理,比如图~\ref{fig:subfig-a} 和图~\ref{fig:subfig-b}。
%
%\begin{figure}[h]
%  \centering
%  \begin{subfigure}{0.2\textwidth}
%    \includegraphics[width=\linewidth]{whulogo.pdf}
%    \caption{武汉大学校徽}
%    \label{fig:subfig-a}
%  \end{subfigure}\qquad
%  \begin{subfigure}{0.7\textwidth}
%    \includegraphics[width=\linewidth]{whu.pdf}
%    \caption{武汉大学}
%    \label{fig:subfig-b}
%  \end{subfigure}
%  \caption{多个分图的示例}
%  \label{fig:multi-image}
%\end{figure}
%
%\section{表格的使用}
%表格的输入可能会比较麻烦,可以使用在线的工具,如 \href{https://www.tablesgenerator.com/}{Tables Generator} 能便捷地创建表格,也可以使用离线的工具,如 \href{https://ctan.org/pkg/excel2latex}{Excel2LaTeX} 支持从 Excel 表格转换成 \LaTeX{} 表格。\href{https://en.wikibooks.org/wiki/LaTeX/Tables}{LaTeX/Tables} 上及 \href{https://www.tug.org/pracjourn/2007-1/mori/mori.pdf}{Tables in LaTeX} 也有更多的示例能够参考。
%
%\subsection{普通表格}
%下面是一些普通表格的示例:
%
%\begin{table}[ht]
%  \centering
%  \caption{简单表格}
%  \label{tab:1}
%  \begin{tabular}{|l|c|r|}
%    \hline
%    我是 & 一只 & 普通 \\
%    \hline
%    的   & 表格 & 呀   \\
%    \hline
%  \end{tabular}
%\end{table}
%
%也可以使用 booktabs 包创建三线表。
%
%\begin{table}[ht]
%  \centering
%  \caption{一般三线表}
%  \label{tab:2}
%  \begin{tabular}{ccc}
%    \toprule
%    姓名 & 学号 & 性别 \\
%    \midrule
%    张三 & 001  & 男   \\
%    李四 & 002  & 女   \\
%    \bottomrule
%  \end{tabular}
%\end{table}
%
%三线表中三条横线分别使用 \verb|\toprule|、\verb|\midrule| 与 \verb|\bottomrule|。若要添加 \(m\)--\(n\) 列的横线,可使用 \verb|\cmidrule{m-n}| 。
%
%要创建占满给定宽度的表格需要使用到 tabularx 包提供的 tabularx 环境。引用表格与其它引用一样,只需要如表~\ref{tab:3}。
%
%\begin{table}[ht]
%  \centering
%  \caption{占满文字宽度的三线表}
%  \label{tab:3}
%  \begin{tabularx}{\textwidth}{CCCC}
%    \toprule
%    序号 & 年龄 & 身高   & 体重  \\
%    \midrule
%    1    & 14   & 156    & 42    \\
%    2    & 16   & 158    & 45    \\
%    3    & 14   & 162    & 48    \\
%    4    & 15   & 163    & 50    \\
%    \cmidrule{2-4} %添加2-4列的中线
%    平均 & 15   & 159.75 & 46.25 \\
%    \bottomrule
%  \end{tabularx}
%\end{table}
%
%\subsection{跨页表格}
%跨页表格常用于附录(把正文懒得放下的实验数据统统放在附录的表中)。一般使用 longtable 包提供的 longtable 环境。若要要创建占满给定宽度的跨页表格,可以使用 xltabular 包提供的 xltabular 环境,使用方法与 longtable 类似。以下是一个文字宽度的跨页表格的示例:
%
%\begin{xltabular}{\textwidth}{CCCCCCCCC}
%  \caption{文字宽度的跨页表格示例}  \\
%  \toprule
%  1 & 0 & 5 & 1 & 2 & 3 & 4 & 5 & 6 \\
%  \midrule
%  \endfirsthead
%
%  \multicolumn{9}{l}{接上一页}      \\
%  \toprule
%  1 & 0 & 5 & 1 & 2 & 3 & 4 & 5 & 6 \\
%  \midrule
%  \endhead
%
%  \toprule
%  \multicolumn{9}{r}{转下一页}
%  \endfoot
%
%  \bottomrule
%  \endlastfoot
%
%  1 & 0 & 5 & 1 & 2 & 3 & 4 & 5 & 6 \\
%  1 & 0 & 5 & 1 & 2 & 3 & 4 & 5 & 6 \\
%  1 & 0 & 5 & 1 & 2 & 3 & 4 & 5 & 6 \\
%  1 & 0 & 5 & 1 & 2 & 3 & 4 & 5 & 6 \\
%  1 & 0 & 5 & 1 & 2 & 3 & 4 & 5 & 6 \\
%  1 & 0 & 5 & 1 & 2 & 3 & 4 & 5 & 6 \\
%  1 & 0 & 5 & 1 & 2 & 3 & 4 & 5 & 6 \\
%  1 & 0 & 5 & 1 & 2 & 3 & 4 & 5 & 6 \\
%  1 & 0 & 5 & 1 & 2 & 3 & 4 & 5 & 6 \\
%  1 & 0 & 5 & 1 & 2 & 3 & 4 & 5 & 6 \\
%  1 & 0 & 5 & 1 & 2 & 3 & 4 & 5 & 6 \\
%  1 & 0 & 5 & 1 & 2 & 3 & 4 & 5 & 6 \\
%  1 & 0 & 5 & 1 & 2 & 3 & 4 & 5 & 6 \\
%  1 & 0 & 5 & 1 & 2 & 3 & 4 & 5 & 6 \\
%  1 & 0 & 5 & 1 & 2 & 3 & 4 & 5 & 6 \\
%  1 & 0 & 5 & 1 & 2 & 3 & 4 & 5 & 6 \\
%  1 & 0 & 5 & 1 & 2 & 3 & 4 & 5 & 6 \\
%  1 & 0 & 5 & 1 & 2 & 3 & 4 & 5 & 6 \\
%  1 & 0 & 5 & 1 & 2 & 3 & 4 & 5 & 6 \\
%  1 & 0 & 5 & 1 & 2 & 3 & 4 & 5 & 6 \\
%\end{xltabular}
%
%
%\section{列表的使用}
%下面演示了创建有序及无序列表,如需其它样式,\href{https://www.latex-tutorial.com/tutorials/lists/}{LaTeX Lists} 上有更多的示例。
%
%\subsection{有序列表}
%这是一个计数的列表
%\begin{enumerate}
%  \item 第一项
%        \begin{enumerate}
%          \item 第一项中的第一项
%          \item 第一项中的第二项
%        \end{enumerate}
%  \item 第二项
%        \begin{enumerate}[label=(\roman*)]
%          \item 第一项中的第一项
%          \item 第一项中的第二项
%        \end{enumerate}
%  \item 第三项
%\end{enumerate}
%
%\subsection{不计数列表}
%这是一个不计数的列表
%\begin{itemize}
%  \item 第一项
%        \begin{itemize}
%          \item 第一项中的第一项
%          \item 第一项中的第二项
%        \end{itemize}
%  \item 第二项
%  \item 第三项
%\end{itemize}
%
%\begin{table}[b]
%  \caption{模板定义的数学环境}\label{tab:数学环境}
%  \begin{tabularx}{\textwidth}{CCCC}
%    \toprule
%    theorem     & definition & lemma  & corollary \\
%    定理        & 定义       & 引理   & 推论      \\
%    \midrule
%    proposition & example    & remark & proof     \\
%    性质        & 例         & 注     & 证明      \\
%    \bottomrule
%  \end{tabularx}
%\end{table}
%
%\section{数学环境的使用}
%模板简单定义了 8 种数学环境,具体见表~\ref{tab:数学环境},使用方法如下所示。
%
%\begin{theorem}
%  设向量 \(\vb*{a} \neq \vb*{0}\),那么向量 \(\vb*{b} \parallel \vb*{a}\) 的充分必要条件是:存在唯一的实数 \(\lambda\),使 \(\vb*{b} = \lambda \vb*{a}\)。
%\end{theorem}
%\begin{definition}
%  这是一条定义。
%\end{definition}
%\begin{lemma}
%  这是一条引理。
%\end{lemma}
%\begin{corollary}
%  对数轴上任意一点 \(P\),轴上有向线段 \(\overrightarrow{OP}\) 都可唯一地表示为点 \(P\) 的坐标与轴上单位向量 \(\vb*{e}_u\) 的乘积:\(\overrightarrow{OP} = u \vb*{e}_u\)。
%\end{corollary}
%\begin{proposition}
%  这是一条性质。
%\end{proposition}
%\begin{example}
%  这是一条例。
%\end{example}
%\begin{remark}
%  这是一条注。
%\end{remark}
%\begin{proof}
%  留作练习。
%\end{proof}
%
%若要定义自己的数学环境,可通过如下代码实现:
%\begin{verbatim}
%\newtheorem{nonsense}{胡说}
%\newtheorem*{bullshit}{八道}
%\end{verbatim}
%其中,带星号 * 的命令不会自动编号。
%
%\newtheorem{nonsense}{胡说}
%\newtheorem*{bullshit}{八道}
%
%\begin{nonsense}
%  啊吧啊吧啊吧。
%\end{nonsense}
%
%\begin{bullshit}
%  不啦不啦不啦。
%\end{bullshit}
%
%\section{单位}
%单位的输入请使用 siunitx 包中提供的 \verb|\si| 与 \verb|\SI| 命令,可以方便地处理希腊字母以及数字与单位之间的空白。在以前,\LaTeX{} 中输入角度需要使用 \verb|$^\circ$| 的奇技淫巧,现在只需要 \verb|\ang| 命令解决问题。当然 siunitx 包中还提供了不少其他有用的命令,有需要的可以自行阅读 siunitx 文档。
%
%示例:\SI{6.4e6}{m},\SI{9}{\micro\meter},\si{kg.m.s^{-1}},\ang{104;28;}。
%
%\section{物理符号}
%physics 宏包可以让用户更加方便、简洁地使用、输入物理符号,具体也请自行阅读 physics 文档。示例如下
%\begin{equation}
%  \begin{aligned}
%    \int_0^{2\symup{\pi}} \abs{\sin{x}} \dd{x} & = 2 \int_0^{\symup{\pi}} \sin{x} \dd{x} \\
%                                                      & = -2 \eval{\cos{x}}_0^{\symup{\pi}}     \\
%                                                      & = 4
%  \end{aligned}
%\end{equation}