%Chapter 6

\chapter{结论与启示}
本文基于2002-2019年的季度财务报表数据,选取缺失值比例低于5\%的51个财务指标,采用BRTs模型、LightGBM算法完成了对公司内在价值的预测,并以估计的内在价值与市场价值的差异构建了错误定价因子$M$,当$M$为负值时,说明公司价值最被高估,当$M$为正值时,说明公司价值最被低估。根据错误定价因子$M$对公司进行分类,并进行相应的高卖低买策略,研究证明从最被高估到最被低估的投资组合,超额收益$\alpha$呈现显著的递增关系;并且其多空组合产生了显著的正收益,但收益保持下降关系,这也强调了信息迭代的重要性,需经常性对数据进行重新调整与分析。以上结论在Fama-MacBeth截面回归和季度投资组合的五种因子模型检验中均显著符合预期,能够通过未来36期的稳健性测试,具有可观的经济意义。
综上所述,本文的实证结果表明,基于季度财务报表的统计分析能够有效运用于中国A股市场,基本面分析起到了一定的预测作用。

总体来说,BRTs模型在本文的股票价格预测、错误定价情况与资产组合配置研究中表现良好,进一步丰富和拓展了该领域下的实证研究结果。与传统的线性研究相比,基于机器学习算法的统计方法有着以下几个特点:
\begin{itemize}
\item[(1)] 本文所采用的数据包含了从02年初到19年底的绝大多数A股市场股票,并且选取了51个财务指标进行预测,整体数据体量更大,维度更高。与经典的线性回归方法相比,机器学习算法数据分析能力更为优越,并且对于高维数据的处理更为谨慎。而且本文采用的BRTs模型比线性模型更易让人理解背后的原理,也符合常人的思维方式;同样,树模型对于缺失值并不敏感,而线性模型会去掉有缺失值的整行数据,对于数据整体结构的影响太大。
\item[(2)] 运用机器学习方法对中国A股市场资产定价与配置的研究现在仍很少,但该领域对于未来金融研究领域的发展有着重要的铺垫作用,因为其将金融与计算机结合到了一起,成功将计算机科学的逻辑思维运用到了金融领域的数据分析问题上,为后续研究提供了大量的相关实证研究结果\cite{zhaoJiQiXueXiZaiJinRongZiChanJieGeYuCeHePeiZhiZhongDeYingYongYanJiuShuPing2020}。一方面,随着机器学习算法的不断优化,提高了对于资产价格预测的准确性,同样让资产配置有了更多可能的组合。但另一方面,机器学习算法的“黑匣子”特征也会影响具体问题的结果,因为输入变量与输出结果可能很难具有逻辑上的联系,在研究过程中我们更关心实证的结果而非理论,机器学习方法可能会限制问题的深入理解。所以,要做到将机器学习方法有效融入到金融领域的研究中,仍是未来需要解决的一个关键问题,需要做到两者的平衡。
\end{itemize}

%不足


基本面分析的有效性也反映出我国资本市场的效率低下,虽然我国A股市场总市值近80万亿元,已发展为全球第二大股票市场。但仍然由于成立时间晚、相关制度不完善、散户比例大、交易成本高等问题,导致了定价效率的低下\cite{wangJiBenMianFenXiZaiZhongGuoAGuShiChangYouYongMaLaiZiJiDuCaiWuBaoBiaoDeZhengJu2018}。而市场效率的提高需要从多方面努力。
\begin{itemize}
\item[(1)] 投资者对于市场信息应仔细甄别,避免盲目跟风,根据自身的风险承担水平构建适合自己的投资组合;提高认知水平,学习相关财务知识,能够做到财务报表的简单阅读,市场基本面信息的理解。

\item[(2)] 监管部门应加强建立与完善监管披露机制,让信息传递更具透明性与高效率。在现有科技社会,推进市场数据的信息化,利用大数据技术做到数据的及时披露,及时筛选并纠正市场上的错误信息,引导投资者理性投资;相反,该举措也能做到信息的向上反馈,让政策制定者更为清晰的了解到市场的实际运行情况,为国家宏观政策的制定提供依据。但同时也要做到对用户个人隐私的保护。

\item[(3)] 上市公司需严格遵守会计规则,做到财务报表的及时、准确、公平披露,不得误导投资者,并做好内幕信息的知情人登记工作,减少投资者层面上的信息不对称。


\end{itemize}


