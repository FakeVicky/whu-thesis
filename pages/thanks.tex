% 致谢页

回想起在武汉的七年生活,总会感觉到不可思议,感谢幸运之神的眷顾。

从初三开始,我对自己的定位就非常清晰:不是那种整天只会学习的那种书呆子,是特别喜欢上网冲浪紧跟潮流的那种“不良少女”,娱乐活动是我生活中不可或缺的一部分。

人生最大的转折点便是2014年,在没有竞赛基础的情况下误打误撞过了华师一附中的专县生考试。在考试的前两晚,我爸在家里拿着在淘宝上买的华师一专县生真题给我讲到了凌晨两点,还记得考数学的时候,选择题都是靠代入选项猜的,后面一看竟然都蒙对了。妈妈在食堂等我的时候遇到了一群从宜昌城区来考试的家长们,但他们都是学校统一培训过,并且组织带队过来的,所以也瞧不起从县城过来的我们。

高中开始,妈妈和大姨一起到武汉来工作和陪读,误打误撞赶上了好时候,自此在武汉有了真正意义上的家。特别感谢我的父母,成功从小县城跃居到了武汉,高中和大学期间没有让我吃过亏,在花钱方面从不吝啬,也从不限制我的发展方向。

之后便是三年丰富多彩的高中生活,感谢我的母校华师一附中,让我深刻体会到了“把时间还给学生,把方法教给学生”的教育模式,在校期间大多数时间还是非常快乐的,以至于到现在都能受益。另一方面,感谢优秀的高中同学们,都是超高质量的人脉资源。班上一共有7个人在武大,可以经常聚会或是喊着跑腿,大学期间从未孤独。

感谢高二的同桌陈同学,整个高三生活都因为我们而变成了彩色,怀念总是给来很晚的你打水的早晨,拿现金去吃洪湖人家铁板牛蛙的中午,或是一起走敏行环路吃外卖的晚上。虽然天各一方,未来过去我只想见你。

妈妈在怀孕的时候梦到在樱花树下打滚,或许冥冥之中注定着我要来武汉大学。很庆幸自己在填志愿时还是选择了武大,有着体面的专业,在学生工作和课外活动方面都很尽兴。不得不感叹在武汉读书简直太方便了,坐上牌坊的586就可以回家,说着湖北塑普也没有突兀的感觉,还有好多在武汉读书的高中同学的陪伴。

当然,最感谢的还是本科期间认识的老师和同学们。感谢两个辅导员以及班导马亮老师,帮助我作为班长做好1703班的学生工作;感谢刘岩老师,在大三的时候跟随参与了因果推断研讨班,参加了CBD数据库的建设工作,免费蹭了很多饭,领了工资,在指导下完成了保研论文,以及无数的推荐信;感谢李斌老师,同样也是无数的推荐信,大四进组蹭了好多次活动,入门了量化投资领域,加入了金融科技研讨班,并完成了这篇毕业论文;最后感谢我可爱的大学同学们,一起吃了好多好多顿饭,特别是金工姐妹花孔彤阳同学,一起参加了很多比赛,八了很多卦,合作了很多项目,以及硕士阶段能一起去到人大。

我是一个很有仪式感的人,但也是一个多愁善感很爱哭的人,从小到大,只要遇到很令人激动或者感动的事情都会忍不住掉眼泪,甚至每次和别人吵架的时候也会忍不住哭,我妈说是我太过于娇气。
由于华师一并不是高考考点,大家都被分到了水高、十四中等学校,在高考前几天便放假了,大家准备回家进行最后的冲刺,没有人说再见。
还记得当时我用教室上面的电脑播放着closer,一边唱一边收拾着东西,到了要走的时候,教室里的人已经寥寥无几,还剩下坐在角落的梁同学,便跑过去抱着她一边哭一边说高考加油呀。
现在就能预想到毕业典礼当天我穿着学士服不知道抱着哪个同学失声痛哭的样子。

写完论文便是毕业,便是和一段人生告别。在追梦过程中,我获得的已经比梦想本身更多,下一站——北京见。